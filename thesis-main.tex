\documentclass[12pt,a4paper,openright,twoside]{book}
\usepackage[utf8]{inputenc}
\usepackage{disi-thesis}
\usepackage{code-lstlistings}
\usepackage{notes}
\usepackage{shortcuts}
\usepackage{acronym}

\school{\unibo}
\programme{Corso di Laurea Triennale in Ingegneria e Scienze Informatiche}
\title{Decentralized Messaging System with Aggregate Computing}
\author{Luca Marchi}
\date{\today}
\subject{Programmazione ad Oggetti}
\supervisor{Prof. Pianini Danilo}
\cosupervisor{Dott.ssa Cortecchia Angela}
% \morecosupervisor{Dott. CoSupervisor 2}
\session{II}
\academicyear{2024-2025}

% Definition of acronyms
\acrodef{IoT}{Internet of Thing}
\acrodef{vm}[VM]{Virtual Machine}


\mainlinespacing{1.241} % line spacing in mainmatter, comment to default (1)

\begin{document}

\frontmatter\frontispiece


\begin{abstract}	

\end{abstract}

\begin{dedication} % this is optional
To my family.
\end{dedication}

%----------------------------------------------------------------------------------------
\tableofcontents   
\listoffigures     % (optional) comment if empty
\lstlistoflistings % (optional) comment if empty
%----------------------------------------------------------------------------------------

\mainmatter

%----------------------------------------------------------------------------------------
\chapter{Introduction}
\label{chap:introduction}
%----------------------------------------------------------------------------------------
\section{Context}
In a world where Internet of Things (\ac{IoT}) devices are becoming increasingly prevalent,
    the need for efficient and reliable communication systems is paramount
    \begin{figure}
        \centering
        \fbox{\includegraphics[width=.8\linewidth]{figures/iot-devices.png}}
        \caption{A variety of IoT devices communicating with each other.}
        \label{fig:iot-devices}
    \end{figure}
    \Cref{fig:iot-devices}.
    The interactions between neighboring devices play a crucial role in enabling seamless data exchange and coordination,
    so there is the need to design networks with infrastructures that support scalability, adaptability and reusability \cite{beal2015aggregate}.
    In the past, it was reasonable to use a programming model that focused on the
    individual computing device, and its relationship with one or more users.
    However, as systems have grown in scale and complexity with the number of computing devices rising,
    this method has become inadequate.
    Traditional network architectures, rely heavily on centralized
    infrastructures, making them unsuitable for scenarios such as disaster recovery or interactions with neighboring devices.
    The computational model of \textit{aggregate computing} provides a promising
    approach to address these challenges by enabling decentralized and self-organizing systems \cite{viroli2019aggregate}.
    Building on this concept, this thesis explores how aggregate programming
    can support a proximity and decentralized messaging system in a network.



\subsection{Aggregate Programming}
\textbf{Aggregate programming} is a distributed systems paradigm that simplifies programming large networks of devices by focusing on the global, system-level behavior rather than the individual behavior of each device.


\chapter{State of the art}


\chapter{Contribution}

You may also put some code snippet (which is NOT float by default), eg: \cref{lst:random-code}.

\lstinputlisting[float,language=Java,label={lst:random-code}]{listings/HelloWorld.java}

\section{Fancy formulas here}

%----------------------------------------------------------------------------------------
% BIBLIOGRAPHY
%----------------------------------------------------------------------------------------

\backmatter


\bibliographystyle{alpha}
\bibliography{bibliography}

\end{document}
